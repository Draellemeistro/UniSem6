\subsection{Elaborating on the function of roles}
% MAC: Mandatory Access Control og DAC: Discretionary Access Control
\subsubsection{RBAC: Role-Based Access Control}
Many large organizations implement Role-based Access Control in one aspect or another\citep{RoleEngMethStd}, doubly so for systems primarily angled at services\citep{FormNormRBAC}
\subsubsection{ABAC: Attribute-Based Access Control}

\subsubsection{ACL: Access Control List}
\subsubsection{DAC: Discretionary Access Control}
\subsubsection{MAC: Mandatory Access Control}
\subsubsection{RuBAC: Rule-Based Access Control}
\subsubsection{RiskBAC: Risk-Based Access Control}
\subsubsection{PBAC: Policy-Based Access Control}
\subsubsection{GBAC: Graph-Based Access Control}

With the most notable, in relation to IAM, being Role-Based Access Control (RBAC), as it is widely used in the industry\citep{Gartner-DefIAM}.

Role Based Access Control (RBAC) has become a de-facto standard to control the access to restricted resources in complex systems and is widely deployed in many commercially available applications, including operating systems, databases and other software

\subsection{Complexity of Role-Based Methods}
The addition of roles allows for a more fine-grained control of access, but also adds complexity to the system. If the roles haven't already been defined, the process of defining them can become a sizable time investment that rapidly increases along the size of the organization\citep{FormNormRBAC2009}. Beyond that, the whole concept of roles and their implementation hasn't been stardized, leading to a wide variety of implementations and interpretations of the concept\citep{RolesInInfoSec2011}.

\subsubsection{Role Engineering}
The problem of defining the roles in a system, has lead to the establishment of the Role Engineering field\citep{RoleEngMethStd}. Role Engineering entails the process of defining roles, and their respective sets of permissions, in a system. This process can be done in a variety of ways, and different methodologies exist\citep{RoleEngMethStd}.




\section{Adding Attributes to Role-Based Access Control}
%https://csrc.nist.rip/groups/sns/rbac/documents/kuhn-coyne-weil-10.pdf
A pure RBAC solution may have inadequate support for dynamic attributes such as time of day, which may need to be considered in determining user permissions. To support dynamic attributes, particularly in large organizations, a role explosion problem may result where thousands of separate roles are needed for different collections of permissions. Recent interest in attribute-based access control (ABAC) suggests that attributes and rules could either replace RBAC or make RBAC more simple and flexible.
------------------------------------------------------------
% %%%%%%% %%%%%%%  %%%%%%%  %%%%%%%  %%%%%%%  %%%%%%%  %%%%%%%
% %%%%%%% %%%%%%%  %%%%%%%  %%%%%%%  %%%%%%%  %%%%%%%  %%%%%%%


